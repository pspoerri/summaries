\newpage
\part{Clustering}

\section{Introduction}
A set of datapoints in a $d$-dimensional Euclidean space is given.
\begin{description}
    \item[Aim] The aim is to find a \emph{meaningful partition} of the data; i.e. label each data point with a unique value $\{1,\ldots, k\}$.
    \item[Objective] The partition should group together similar data points, while the different groups/clusters should be as dissimilar as possible from each other.
    \item This way we can uncover similarities between data points and give rise to data compression schemes.
\end{description}


\subsection{Problem}
Consider $N$ data points in a $D$-dimensional space. Each data vector is denoted by $x_n$, $n=1,\ldots,N$. Our goal is to partition the data set into $K$ clusters: Find vectors $u_1, \ldots, u_K$ that represent the centroid of each cluster.

A datapoint $x_n$ belongs to cluster $k$ if the Euclidean distance between $x_n$ and $u_k$ is smaller than the distance to any any other centroid.

Mathematically, the clustering problem defines a mixed discrete continuous optimisation problem. 

\subsubsection{The Cost Function of Vector Quantisation}
\begin{description}
    \item[Objective] Minimise the cost function
        \begin{align*}
            J(U,Z) = \norm{X-UZ}_F^2 = \sum_{n=1}^N\sum_{k=1}^K z_{k,n} \norm{x_n-u_k}_2^2
        \end{align*}
        where
        \begin{align*}
            X &= [x_1,\ldots, x_N] \in \R^{D\times N}\\
            U &= [u_1,\ldots, u_K] \in \R^{D\times K}, &\text{centroids}\\
            Z &\in \{0,1\}^{K\times N}, &\text{assignments}
        \end{align*}
        with $\sum_k z_{k,n} = 1 \forall n$ i.e.,  one element per columns set to $1$.
\end{description}
Assignment notation:
\begin{description}
    \item[Assignment Notation]: Vector $\hat z \in \{1,\ldots,K\}^N$ indicating for each data point to which cluster index it is assigned:
    \begin{align*}
        \hat z = \begin{pmatrix}
                    2\\ 3\\ 4\\ 2 \\ 2\\ 1
                 \end{pmatrix}
    \end{align*}
    \item[Matrix Notation]: The matrix $Z\in \{0,1\}^{K\times N}$ with only one non-zero entry per column, assigns data points to clusters:
    \begin{align*}
        Z &= \begin{pmatrix}
                 0&0&0&0&0&1\\
                 1&0&0&1&1&0\\
                 0&1&0&0&0&0\\
                 0&0&1&0&0&0
             \end{pmatrix}
    \end{align*}
\end{description}
