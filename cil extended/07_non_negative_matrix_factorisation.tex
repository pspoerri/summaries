\section{Non-Negative Matrix Factorisation}

Problem:
\begin{itemize}
    \item Given: Corpus of text documents such as web pages
    \item Goal: Find a low-dimensional representation of these documents. 
\end{itemize}

\paragraph{Document Representation}
\begin{description}
\item[Vocabulary] Every semantically "useful"word in a language. This excludes all the stop words (the, is, at, which, etc...). Stemming reduces the text to it's root form: Cats, catlike, catty, etc. all map to the same word "cat".


$D$ denotes the size of the vocabulary.

\item[Document] \emph{Bag of words}-model: Ordering of the words in a document is ignored. The document is represented by a vector of length $D$ with frequencies/counts of different words. This vector is usually very sparse.
\end{description}

\subsection{Matrix view} Here $N$ denotes the number of documents and $K$ denotes the number of "clusters" with $D$ being the vocabulary size.
\begin{itemize}
    \item $X \in \R_+^{D\times N}$ denotes the document-term matrix which stores the word counts for each document:
        \begin{align*}
            X = [x_1,\ldots, x_N]
        \end{align*}
        $x_{dn}$: Frequency of the $d$-th word in the $n$-th document.
    \item We study a non-negative matrix factorisation (NMF) of the document matrix $X$:
        \begin{align*}
            X\approx UZ
        \end{align*}
        with $U\in \R_+^{D\times K}$ and $Z\in \R_+^{K\times N}$, with $\R_+ := [0,\ldots, \infty)$ . 
\end{itemize}
\subsection{Methods}
\subsubsection{Full Singular Value Decomposition}
A singular value decomposition can be used to decompose the document word matrix:
\begin{align*}
    X = U\Sigma V^T.
\end{align*}
However the procured $U$ and $V$ are not guaranteed to be non-negative. 

\subsubsection{Classic Latent Semantic Indexing (LSI)}
The LSI method uses a partial SVD
\begin{align*}
    \tilde X = U\tilde \Sigma V^T \approx X.
\end{align*}
With $\tilde \Sigma$ having all but the largest $K$ singular values set to zero. 

\paragraph{Mapping function} A query $x$ can now be mapped by the query can now be compared/queried with an inner product:
\begin{align*}

\end{align*}

