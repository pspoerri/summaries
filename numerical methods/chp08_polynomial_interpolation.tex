\part{Polynomial Interpolation}

\section{Polynomials}
	\[
	 \mathcal P_k := \{ t\mapsto a_k t^k + a_{k-1} t^{k-1} + \cdots + a_0, \ a_j \in \K
	\]

	\begin{theorem}[Dimension of Space of Polynomials]
	 \[
	  \dim \mathcal P_j = k+1 \qquad \text{ and } \qquad \mathcal P_k \subset C^\infty (\R)
	 \]
	\end{theorem}
	
	Matlab:
	\[
	 a_k t^k + a_{k-1} t^{k-1} + \cdots + a_0 \quad \text{(use horner schema to calculate)}
	\]
	\begin{lstlisting}
	 polyval(p,x);
	\end{lstlisting}
\section{Polynomial Interpolation: Theory}
	Given: Simple nodes $t_0, \ldots, t_n$, $n \in \N$, $-\infty < t_0 < t_1 < \cdots < t_n < \infty$ and the values $y_0, \ldots, y_n \in \K$ compute $p \in \mathcal P_n$ such that
		\[
		 p(t_j) = y_j \qquad \text{for }j=0, \ldots n
		\]

	\subsection{Lagrange Polynomials}
		For \emph{nodes} $t_0 < t_1 < \cdots < t_n$ consider
		\[
		 \textbf{Lagrange Polynomials: }\ L_i (t) = \prod_{j=0\ \land \ j \neq i}^n \frac{t-t_j}{t_i-t_j}
		\]

\section{Chebychev Interpolation}
	\begin{definition}[Chebychev Polynomial]
	 The $n^{th}$ \emph{Chebychev Polynomial} is 
	 \begin{align*}
		T_n (t) &= \cos(n \arccos(t)) \qquad -1 \leq t \leq 1\\
		\text{Zeros of }T_n: \qquad t_k &= \cos \left( \frac{2k-1}{2n}\pi\right),\quad k=1, \ldots, n	  
	 \end{align*}
	\end{definition}
	
	\paragraph{Scaling argument}
	\[
	 [-1,1] \xrightarrow{\hat t \mapsto t:= a + \frac{1}{2}(\hat t+1)(b-a) } [a,b] \qquad \hat f(\hat t) := f(t)
	\]

	\subsection{Computational Aspects}
		\begin{theorem}[Orthogonality of Chebychef polynomials]
			The Chebychef polynomials are orthogonal with respect to the scalar product
			\[
			 \left< f,g \right> = \int_{-1}^1 f(x) g(x)\frac{1}{\sqrt{1-x^2}} dx
			\]		 
		\end{theorem}

		\begin{theorem}[Representation formula]
		 The interpolation polynomial $p$ of $f$ in the Chebychev nodes $x_0, \ldots, x_n$ (the zeros of $T_{n+1}$ is given by:
		 \[
		  p(x) = \frac{1}{2}c_0 + c_1 T_1(x) + \ldots c_n T_n(x)
		 \]
		 with
		 \[
		  c_k = \frac{2}{n+1}\sum_{t=0}^n f \left( \cos\left(\frac{2l+1}{n+1}\cdot \frac{\pi}{2}\right) \right) \cdot \cos \left( k\frac{2l+1}{n+1}\cdot \frac{\pi}{2}\right)
		 \]
		\end{theorem}
		
		\begin{theorem}[Clenshah algorithm]
		Let $p\in \mathcal P_n$ be an arbitrary polynomial
		\[
		 p(x) = \frac{1}{2}c_0 +c_1 T_1(x) +\ldots + c_n T_n(x)
		\]
		 Set
		 \begin{align*}
		  d_{n+2} &= d_{n+1} = 0\\
		  d_k &= c_k+(2x)\cdot d_{k+1}-d_{k+2} \qquad \text{for } k=n, n-1,\ldots,0
		 \end{align*}
		 Then
		 \[
		  p(x) = \frac{1}{2}(d_0-d_2)
		 \]
		 The Clenshaw algorithm is numerically stable
		\end{theorem}

