\section{Cache Coherence}
Processor speed improves (approx.) 60\% per year:
\begin{itemize}
	\item until 2005 mainly by increasing the clock rate,
	\item after 2005 by increasing the number of cores.
\end{itemize}
The memory speed however only improves by (approx.) 10\% per year. Which results in a widening performance gap.
 
Caching may result in multiple copies for a memory location in multiple caches. Cache coherence manages the existence of multiple copies. 
\emph{Requirements}: 
\begin{itemize}
	\item Write propagation: Writes are eventually visible to all processors
	\item Write serialisation: Every processor should see the writes to the same location in the same order.
\end{itemize}

\begin{description}
	\item[Concerns] $\text{}$
		\begin{itemize}
			\item Performance
			\item Implementation cost
			\item Correctness
		\end{itemize}
	\item[Issues] $\text{}$
		\begin{itemize}
			\item Detection (when is it necessary for the cache controller to become active)
			\item Enforcement (how the coherence is established)
			\item Precision of block-sharing information
			\item Block size
		\end{itemize}
	\item[Mechanisms] $\text{}$
		\begin{description}
			\item[Snooping-based cache coherence] $\text{}$
				\begin{itemize}
					\item Shared bus or network
					\item Cache controller "snoops" (observes) bus transactions
					\item Cache controller aware of state of the cache's data
				\end{itemize}

			\item[Directory-based cache coherence] Record (in memory) information necessary to maintain coherence
		\end{description}

\end{description}

\subsection{Problems}
\begin{enumerate}
	\item Stale values
		\begin{itemize}
			\item $CPU_1$ uses stale value that has already been modified by $CPU_0$
			\item \emph{Solution} 
				\begin{itemize}
					\item Don't allow this to happen
					\item Invalidate all copies before allowing a write to proceed
				\end{itemize}
		\end{itemize}
	\item Loosing result of a write
		\begin{itemize}
			\item An incorrect write-back order of modified cache lines applies the writebacks in an order different from the order of the write operations
			\item \emph{Solution}> Disallow more than one modified copz
		\end{itemize}
\end{enumerate}

\subsection{Protocol Approaches}
\begin{description}
	\item[Invalidation-based] $\text{}$
		\begin{itemize}
			\item All coherency-related traffic broadcast to all CPUs
			\item Each processor snoops traffic and reacts accordingly
				\begin{itemize}
					\item Invalidate lines written to by antother CPU
					\item Signal sharing for cache lines currently in cache
				\end{itemize}
			\item Straightforward solution for bus-based systems
			\item Suited for small-scale systems
			\item Comparison:
				\begin{itemize}
					\item Only write misses hit the bus (suited for write-back caches)
					\item Subsequent writes to the cache-line are write-hits
					\item Good for multiple writes to the same cache line by the same CPU
				\end{itemize}

		\end{itemize}
	\item[Update-based] $\text{}$
		\begin{itemize}
			\item Uses central directory for cache line ownershop
			\item Write operation updates copies in other caches
				\begin{itemize}
					\item Can update all other CPUs at once (less bus traffic)
					\item But: Multiple writes cause multiple updates (more bus traffic)
				\end{itemize}
			\item Suited for large-scale systems
			\item Comparison:
				\begin{itemize}
					\item All sharers of a cache line continue to hit in the cache after a write by one cache.
						\emph{Assumption}: There are more accesses until this line must be evicted for other reasons.
					\item Well-suited for producer-consumer pattern
					\item Risks wasting bandwidth
				\end{itemize}

		\end{itemize}
\end{description}

\subsection{MESI Protocol}
Each cache line is in one of $4$ states>
\begin{description}
	\item[Modified (M)] $\text{}$
		\begin{itemize}
			\item No copies in other caches, local copy has been modified
			\item Memory is stale
		\end{itemize}
	\item[Exclusive (E)]
		\begin{itemize}
			\item No copies of other caches
			\item Memory is up to date
		\end{itemize}
	\item[Shared]
		\begin{itemize}
			\item Unmodified copies \emph{may} exist in other caches
			\item Memorz is up to date
		\end{itemize}

	\item[Invalid (I)] not in cache
\end{description}


